\documentclass[a4paper, 10pt, legalpaper]{exam}
\usepackage[utf8]{inputenc}
\usepackage[spanish]{babel}
\usepackage[margin=.75in]{geometry}
\usepackage{amsmath,amssymb}
\usepackage{multicol}
\usepackage{graphicx}
\usepackage{tikz}

% Configuración del Encabezado
\newcommand{\class}{}
\newcommand{\term}{Métodos Numéricos I}
\newcommand{\examnum}{Primer Examen Parcial}
\newcommand{\examdate}{Fecha:23 de octubre de 2024}
\newcommand{\timelimit}{80 minutos}
\pagestyle{head}
\firstpageheader{Escuela Superior de Física y Matemáticas}{}{Instituto Politécnico Nacional}
\runningheader{\class}{\examnum\ - Page \thepage\ of \numpages}{\examdate}
\runningheadrule

% Configuración de la tabla de calificación
\pointpoints{punto}{puntos}
\hpword{Puntos:}
\vpword{Puntos}
\vtword{Total:}
\htword{Total}
\vsword{Resultado}
\hsword{Resultado:}
\vqword{Problema}
\hqword{Pregunta:}

\begin{document}
	% Definición del Encabezado
	\noindent
	\begin{tabular*}{\textwidth}{l @{\extracolsep{\fill}} r @{\extracolsep{6pt}} l}
		\textbf{\class} & \textbf{Nombre:} & \makebox[2.5in]{\hrulefill}\\
		\textbf{\term} &&\\
		\textbf{\examnum} & \textbf{Escala} & \makebox[2.5in]{\hrulefill}\\
		\textbf{\examdate} &&\\
		\textbf{Tiempo: \timelimit} %& Profesor: & \makebox[2.5in]{\emph{Dr. John Doe}}
	\end{tabular*}\\
	\rule[2ex]{\textwidth}{2pt}
	
	% Bloque de instrucciones
	\noindent
	Este examen contiene \numquestions \;ejercicios que corresponden a \numpoints \;puntos de la primera calificación parcial del curso. Para la calificación de cada ejercicio se consideran los siguientes valores: Planteamineto correcto: 25\%; Desarrollo correcto: 50 \%; Solución correcta (incluye uso correcto de la aproximación): 25\%.
	
	% Tabla de calificaciones
	\begin{center}
		Tabla de calificación de uso exclusivo para el profesor. \\
		\addpoints
		% Puede presentar la tabla de calificación en orientación vertical [v] u horizontal [h]
		\gradetable[h][questions]
	\end{center}
	
	% Preguntas de ejemplo.
	\begin{questions}
		
		\addpoints
		\question[05] Demuestre que la ecuación $x - (lnx)^x = 0$ tiene al menos una solución en el intervalo $[4,5]$.
		
		\addpoints
		\question[05] Demuestre que la primera derivada de la función $ f (x)=1-e^x+(e-1)$ sen $ ((\pi / 2)x)$ se anula al menos una vez en el intervalo $[0,1]$.
		
		
		\addpoints
		\question[10] Sea $f(x)=(x+2)(x+1)x(x-1)^3(x-2)$. ¿En cuál cero de $f$ converge el método de bisección en el intervalo [-3.0, 2.5]?
		
		\addpoints
		\question[10] Aplique el método de iteración de punto fijo para determinar una solución con una exactitud de $10^{-2}$ para $x^3-x-1=0$ en $[1,2]$. Utilice $p_0=1$.
		
		\addpoints
		\question[10] Aplique el método de Newton para obtener una solución con una exactitud de $10^{-4}$ para $x - cosx = 0$ en $[0,\pi/2]$.
		
		\addpoints
		\question[20] Obtenga las aproximaciones, con una exactitud de $10^{-4}$ a todos los ceros del polinomio $P(x) = x^3 - 2x^2-5$ usando el método de Newton.
		
		
	\end{questions}
	
	\noindent
	\rule[2ex]{\textwidth}{2pt}
	\begin{center}
		\textbf{Soluci\'on}
	\end{center}
	
\end{document}